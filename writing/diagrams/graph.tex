\begin{figure}[H]
\centering
\begin{tikzpicture}[shorten >=1pt, -]
  \tikzset{
    every node/.style={%
      draw,
      circle,
      % inner sep=0,
      % outer sep=0,
      minimum size=1.5cm,
      node distance=1cm,
    }
  }
  % \tikzstyle{vertex}=[circle,draw=black,minimum size=12pt,inner sep=2pt]
  
  \node[] (the) {the};
  \node[right=of the] (big) {big};
  \node[right=of big] (dog) {dog};
  \node[right=of dog] (ate) {ate};
  \node[right=of ate] (my) {my};
  \node[right=of my] (steak) {steak};
  \node[above=of the] (the_big) {
    \Tree [ the big ]
  };
  \node[right=of the_big] (big_dog) {
    \Tree [ big dog ]
  };
  \node[above=of the_big] (the_big_dog2) {
    \Tree [ [ the big ] dog ]
  };
  \node[right=of the_big_dog2] (the_big_dog) {
    \Tree [ the [ big dog ] ]
  };
  \node[above=of my] (my_steak) {
    \Tree [ my steak ]
  };
  \node[above=of my_steak] (ate_my_steak) {
    \Tree [ ate [ my steak ] ]
  };

  \draw (the) -- (big) -- (dog) -- (ate) -- (my) -- (steak);
  \draw (the) -- (big_dog);
  \draw (the_big) -- (dog);
  \draw (big_dog) -- (ate);
  \draw (the_big_dog) -- (ate);
  \draw (ate) -- (my_steak);
  \draw (dog) -- (ate_my_steak);
  \draw (big_dog) -- (ate_my_steak);
  \draw (the_big_dog) -- (ate_my_steak);
  \draw [bend left] (the_big_dog2) to (ate_my_steak);

  % \Edge[label=foo] (the_big_dog) (ate);
  % \draw[bend left](the) to (dog);


\end{tikzpicture}
\label{fig:graph}
\caption{A simplified graph representing the utterance `the big dog ate my steak'.
         For display, FTP and BTP edges are represented as a single undirected edge.
         Hierarchical structure is distinctive in this image; however, when
         the model is run in flat mode, the two nodes in the top left would be collapsed
         into a single node.}
\end{figure}